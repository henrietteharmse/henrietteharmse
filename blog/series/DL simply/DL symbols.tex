\documentclass{amsart}
%\documentclass[a4paper,10pt]{scrartcl}

\usepackage[utf8x]{inputenc}
\usepackage[british]{babel}
%\usepackage[a4paper, inner=0.5cm, outer=0.5cm, top=1cm,
%bottom=1.5cm, bindingoffset=1cm]{geometry}
\usepackage{amsmath}
\usepackage{amssymb, latexsym}
\usepackage{longtable}
\usepackage[table]{xcolor}
\usepackage{textcomp} 
\usepackage{graphicx}
\usepackage{enumitem}
\usepackage{hyperref}

\setlist[enumerate]{label*=\arabic*.}
\newtheorem{theorem}{Theorem}[section]
\newtheorem{example}{Example}[section]
\newtheorem{definition}{Definition}[section]
\newtheorem{proposition}{Proposition}[section]
\newtheorem{notation}{Notation}[section]

\title{DL Symbols}
\author{Henriette Harmse}
\date{\today}

\pdfinfo{%
  /Title    (DL Symbols)
  /Author   (Henriette Harmse)
  /Creator  ()
  /Producer ()
  /Subject  (DL)
  /Keywords ()
}

\begin{document}
  \maketitle
  
\begin{longtable}{|>{\footnotesize}p{2cm}|>{\footnotesize}p{1.3cm}|>{\footnotesize}p{4cm}|>{\footnotesize}p{5cm}|}
%%	\rowcolors{1}{white}{lightgray}
%%	\footnotesize
%%	\begin{center} 
	\caption{List of DL Symbols}
		\label{tab_DLSymbols}\\
%%		\smallskip
%%		\begin{tabular}
			\hline
         	\textbf{Name} & \textbf{DL} & \textbf{Manchester} &  \textbf{Semantics} \\
			\hline  
			domain of interpretation & $\triangle^{\mathcal{I}}$ & & The maximal set of all the things this ontology deals with\\
			\hline
			Thing(OWL) or top(DL) & $\top$ & \texttt{owl:Thing} & Equivalent to the domain of interpretation\\
			\hline
			Nothing(OWL) or bottom(DL) & $\bot$ & \texttt{owl:Nothing} & Equivalent to the empty set\\
			\hline
			Top object property & $U$ & \texttt{owl:topObjectProperty} & Maximal set of pairs for the domain of interpretation: $\triangle^{\mathcal{I}} \times \triangle^{\mathcal{I}}$  \\
			\hline
			Bottom object property & $B$ & \texttt{owl:bottomObjectProperty} & Empty set of pairs.  \\
			\hline			
			instance or individual & $c$ & \texttt{Individual: c} & $c \in \triangle^{\mathcal{I}}$ which is read as: $c$ is a member of the domain of interpretation \\
			\hline
			class(OWL) or concept(DL) & $C$ & \texttt{Class: C} & $C \subseteq \triangle^{\mathcal{I}}$, which is read as: $C$ is a set of individuals that is a subset of the domain of interpretation. \\
			\hline
			object property(OWL) or role(DL) & $r$ & \texttt{ObjectProperty: r} & A set of ordered pairs of individuals\\
			\hline
			data property(OWL) or concrete role(DL) & $t$ & \texttt{DataProperty: t} & A set of ordered pairs where the first component of each pair is an individual and the second component is a data value\\
			\hline
			concept subsumption axiom & $C \sqsubseteq D$ & \texttt{Class C:\newline SubClassOf: D} & The set of individuals represented by the concept C is a subset of the set of individuals represented by the concept D\\
			\hline	
			concept equivalence axiom & $C \equiv D$ & \texttt{Class C: \newline EquivalentTo: D} & The set of individuals represented by the concept C is equivalent to the set of individuals represented by the concept D\\
			\hline
			role subsumption axiom & $r \sqsubseteq s$ & \texttt{ObjectProperty r:\newline SubPropertyOf: s} or if r and s are concrete roles/data properties:\newline \texttt{DataProperty r:\newline SubPropertyOf: s} & The set of pairs of individuals represented by the role r is a subset of the set of pairs of individuals represented by the concept s\\
			\hline	
			role equivalence axiom & $r \equiv s$ & \texttt{ObjectProperty r: \newline EquivalentTo: s} & The set of pairs of individuals represented by the role r is equivalent to the set of pairs of individuals represented by the concept s\\
			\hline
			instance of (OWL) or concept assertion (DL) & $C(x)$ & \texttt{Class: C\newline Individual: x \newline Type: C} & The individual x is an instance of the concept C, \\
			\hline
			property assertion (OWL) or role assertion (DL) & $r(x, y)$ & \texttt{ObjectProperty: r\newline Individual: y \newline Individual: x \newline Facts: r y}  &  Individual x is related to individual y via the property r.\\  
			\hline
			individuals are the same & $x \approx y$ & \texttt{Individual: x \newline Individual: y \newline SameAs: x}& \\ 		
			\hline	
			individuals are different & $x \not\approx y$ & \texttt{Individual: x \newline Individual: y \newline DifferentFrom: x} & \\ 		
			\hline					
			concept negation & $\neg C$ & \texttt{not C} & Everything in the domain of interpretation that is not in C \\
			\hline			
			intersection of concepts & $C \sqcap D$ & \texttt{C and D} & The intersection of the sets represented by the concepts C and D. Read as: the conjunction of C and D.\\
			\hline
			disjunction of concepts & $C \sqcup D$ & \texttt{C or D} & The union of the sets represented by the concepts C and D. Read as: the disjunction of C and D.\\
			\hline
			qualified existential restriction & $\exists r.C$ & \texttt{r some C} & This represents the set of individuals such that for each individual d there is at least 1 element e that is linked to an individual d of type C via the role r. Read as: The set of individuals with an r-filler that is of type C.\\
			\hline
			unqualified existential restriction & $\exists r$ or $\exists r.\top$ & \texttt{r some owl:Thing} & This represents the set of individuals such that for each individual d there is at least 1 element e that is linked to an individual d via the role r. Read as: The set of individuals with an r-filler.\\
			\hline
			value restriction & $\exists r.\{x\}$ & \texttt{r hasValue x} & This a more specialized form of the existential restriction. This represents the set of individuals such that for each individual d belonging to the set, it is linked to the individual x (respectively value x when r is a data property or concrete role). Read as: The set of individuals with the individual x (respectively, the value x) as r-filler.\\
			\hline
			qualified universal restriction & $\forall r.C$ & \texttt{r only C} & This is the set of individuals where for each individual d of the set, it holds that whenever d is linked to an individual e via r, then e is of type C. Read as: the set of individuals where all r-fillers are of type C.\\						
			\hline  
			unqualified universal restriction & $\forall r$ or $\forall r.\top$ & \texttt{r only owl:Thing} & This is the set of individuals where for each individual d of the set, it holds that whenever d is linked to an individual e via r, then e is of type C.\\						
			\hline		
			qualified minimum cardinality restriction & $\geq n r.C$ & \texttt{r min n C} & This represents the set of individuals such that each individual d is linked to at least n individuals of type C via the role r. Read as: The set of individuals with at least n r-fillers that are of type C.\\
			\hline
			unqualified minimum cardinality restriction & $\geq n r.\top$ or $\geq n r$ & \texttt{r min n owl:Thing} & This represents the set of individuals such that each individual d is linked to at least n individuals via the role r. Read as: The set of individuals with at least n r-fillers.\\
			\hline	
			qualified maximum cardinality restriction & $\leq n r.C$ & \texttt{r max n C} & This represents the set of individuals such that each individual d is linked to at most n individuals of type C via the role r. Read as: The set of individuals with at most n r-fillers of type C.\\
\hline	
			unqualified maximum cardinality restriction & $\leq n r.\top$ or $\leq n r$ & \texttt{r max n owl:Thing} & This represents the set of individuals such that each individual d is linked to at most n individuals via the role r. Read as: The set of individuals with at most n r-fillers.\\
\hline
			qualified exact restriction & $= n r.C$ & \texttt{r exactly n C} &	This represents the set of individuals such that each individual d is linked to exactly n individuals of type C via the role r. Read as: The set of individuals with exactly n r-fillers of type C.	\\
\hline			
			inverse property(OWL) or inverse role(DL) & $r^-$ & \texttt{ObjectProperty: r \newline ObjectProperty: s \newline InverseOf: r} & The set that is the inverse of r consists of the pairs (x,y) of r that are swapped around to (y,x). \\
			\hline
			property chain(OWL) or role chain(DL) & $p1 \circ p2 \sqsubseteq p$ &  \texttt{ObjectProperty: p1 \newline ObjectProperty: p2 \newline ObjectProperty: p \newline SubPropertyChain: p1 o p2}   & If we have $p1(x,y)$ and $p2(y,z)$, then it follows that $p(x,z)$. An example is \newline \texttt{ObjectProperty: grandParentOf \newline SubPropertyChain: parentOf o parentOf}.\\
			\hline
			TBox & $\mathcal{T}$ & NA & The set of all concept axioms. \\
			\hline
			RBox & $\mathcal{R}$ & NA & The set of all role axioms .\\
			\hline			
			ABox & $\mathcal{A}$ & NA & The set of all assertions regarding individuals. \\
			\hline
			Ontology & $\mathcal{O}$ & NA & Consists of the union of the TBox, RBox and ABox where any of these could be potentially be empty.\\
			\hline				
			entailment & $\vDash$ & NA & States that an axiom or assertion follows from a TBox, RBox, ABox or ontology. I.e., $\mathcal{T} \vDash C \sqsubseteq D$ states that $C$ is subsumed by $D$ follows from the TBox $\mathcal{T}$. Entailment is typically used in discussions wrt reasoning.\\ 
			\hline			
\end{longtable}  
  
 
 
  \bibliographystyle{amsplain}
  \bibliography{../../../BibliographicDetails_v.0.1}
 
\end{document}
