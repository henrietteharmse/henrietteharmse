\documentclass{amsart}
%\documentclass[a4paper,10pt]{scrartcl}

\usepackage[utf8x]{inputenc}
\usepackage[british]{babel}
%\usepackage[a4paper, inner=0.5cm, outer=0.5cm, top=1cm,
%bottom=1.5cm, bindingoffset=1cm]{geometry}
\usepackage{amsmath}
\usepackage{amssymb, latexsym}
\usepackage{longtable}
\usepackage[table]{xcolor}
\usepackage{textcomp} 
\usepackage{stmaryrd}
\usepackage{graphicx}
\usepackage{enumitem}
\usepackage{yfonts}
\usepackage{algpseudocode}
\usepackage{algorithm}
\usepackage{hyperref}
\usepackage{MnSymbol}

\setlist[enumerate]{label*=\arabic*.}
\newtheorem{theorem}{Theorem}[section]
\newtheorem{example}{Example}[section]
\newtheorem{definition}{Definition}[section]
\newtheorem{proposition}{Proposition}[section]
\newtheorem{notation}{Notation}[section]

\renewcommand{\algorithmicrequire}{\textbf{Input:}}
\renewcommand{\algorithmicensure}{\textbf{Output:}}

\title{Why does the Reasoner ignore my Constraint?}
\author{Henriette Harmse}
\date{\today}

\pdfinfo{%
  /Title    (Why does the Reasoner ignore my Constraint?)
  /Author   (Henriette Harmse)
  /Creator  ()
  /Producer ()
  /Subject  (DL)
  /Keywords ()
}

\begin{document}
  \maketitle
  
  
  A most frustrating problem often encountered by people, with experience in relational databases when they are introduced to ontologies, is that ontology reasoners seem to ignore constraints. In this post I give examples of this problem, explain why they happen and I provide ways to deal with each example. 
  
  \section{An Example}
  A typical example encountered in relational databases is that of modeling orders with orderlines, which can be modeled via \texttt{Orders} and \texttt{Orderlines} tables where the \texttt{Orderlines} table has a foreign key constraint to the \texttt{Orders} table. A related ontology is given in Figure~\ref{f_Order}. It creates as expected \texttt{Order} and \texttt{Orderline} classes with a \texttt{hasOrder} object property. That individuals of \texttt{Orderline} are necessarily associated with \textbf{one} order is enforced by \texttt{Orderline} being a subclass of \texttt{hasOrder exactly 1 owl:Thing}. 
  
    \begin{figure}
      %trim option's parameter order: left bottom right top
      \centering \includegraphics[trim = 0mm 0mm 0mm 0mm, clip, scale=0.5]{./Order.png}
      \vspace{-3mm}
      \caption{An ontology modeling orders and orderlines}\label{f_Order}
    \end{figure}
    
  \section{Two Problems}
  Two frustrating and most surprising errors given the Order ontology are: (1) if an \texttt{Orderline} individual is created for which no associated \texttt{Order} individual exists, the reasoner will not give an inconsistency, and (2) if an \texttt{Orderline} individual is created for which two or more \texttt{Order} individuals exist, the reasoner will also not give an inconsistency.
  
  \subsection{Missing Association Problem}
  Say we create an individual \texttt{orderline123} of type \texttt{Orderline}, which is not associated with an individual of type \texttt{Order}, in this case the reasoner will not give an inconsistency. The reason for this is due to the \textbf{open world assumption}. Informally it means that the only inferences that the reasoner can make from an ontology is based on explicit information stated in the ontology or what can derived from explicit stated information. 
  
  When you state \texttt{orderline123} is an \texttt{Orderline}, there is \textbf{no explicit information} in the ontology that states that \texttt{orderline123} is \textbf{not} associated with an individual of \texttt{Order} via the \texttt{hasOrder} property. To make explicit that \texttt{orderline123} is not in such a relation, you have to define \texttt{orderline123} as in Figure~\ref{f_HasNoOrder}. \texttt{hasOrder max 0 owl:Thing} states that it is known that \texttt{orderline123} is not associated with an individual via the \texttt{hasOrder} property.
 
    \begin{figure}
      %trim option's parameter order: left bottom right top
      \centering \includegraphics[trim = 0mm 0mm 0mm 0mm, clip, scale=0.5]{./HasNoOrder.png}
      \vspace{-3mm}
      \caption{Enforcing that \texttt{orderline123} is not associated via \texttt{hasOrder}}\label{f_HasNoOrder}
    \end{figure}
    
  \subsection{Too Many Associated Individuals Problem}
  Assume we now change our definition of our \texttt{orderline123} individual to be associated via \texttt{hasOrder} to two individuals of \texttt{Order} as shown in Figure~\ref{f_HasTwoOrders}. Again, most frustratingly the reasoner does not find that the ontology is inconsistent. The reason for this is that OWL does not make the \textbf{unique name assumption}. This means that individuals with different names can be assumed by the reasoner to represent a single individual. To force the reasoner to see \texttt{order1} and \texttt{order2} as necessarily different, you can state \texttt{order1} different from \texttt{order2} by adding \texttt{DifferentFrom:order2} to \texttt{order1} (or similarly for \texttt{order2}).

     \begin{figure}
      %trim option's parameter order: left bottom right top
      \centering \includegraphics[trim = 0mm 0mm 0mm 0mm, clip, scale=0.5]{./HasTwoOrders.png}
      \vspace{-3mm}
      \caption{\texttt{orderline123} associated with two individuals of type \texttt{Order}}\label{f_HasTwoOrders}
    \end{figure}
    
  \section{Conclusion}
  In this post I explained problems newcomers (particularly with relational database experience) to ontologies may experience and how to deal with them.
  
  
  \bibliographystyle{amsplain}
  \bibliography{../../../BibliographicDetails_v.0.1}
 
\end{document}
