\documentclass{amsart}
%\documentclass[a4paper,10pt]{scrartcl}

\usepackage[utf8x]{inputenc}
\usepackage[british]{babel}
%\usepackage[a4paper, inner=0.5cm, outer=0.5cm, top=1cm,
%bottom=1.5cm, bindingoffset=1cm]{geometry}
\usepackage{amsmath}
\usepackage{amssymb, latexsym}
\usepackage{longtable}
\usepackage[table]{xcolor}
\usepackage{textcomp} 
\usepackage{stmaryrd}
\usepackage{graphicx}
\usepackage{enumitem}
\usepackage{yfonts}
\usepackage{algpseudocode}
\usepackage{algorithm}
\usepackage{hyperref}
\usepackage{MnSymbol}

\setlist[enumerate]{label*=\arabic*.}
\newtheorem{theorem}{Theorem}[section]
\newtheorem{example}{Example}[section]
\newtheorem{definition}{Definition}[section]
\newtheorem{proposition}{Proposition}[section]
\newtheorem{notation}{Notation}[section]

\renewcommand{\algorithmicrequire}{\textbf{Input:}}
\renewcommand{\algorithmicensure}{\textbf{Output:}}

\title{}
\author{}
\date{}

\pdfinfo{%
  /Title    (DBPedia Extraction Framework and Eclipse Quick Start)
  /Author   (Henriette Harmse)
  /Creator  ()
  /Producer ()
  /Subject  (DL)
  /Keywords (Semantic Web, DBPedia Extraction Framework, Eclipse, Scala, Maven)
}

\begin{document}
  \maketitle

  I recently tried to compile the \href{https://github.com/dbpedia/extraction-framework/}{DBPedia Extraction Framework}. What was not immediately clear to me is whether I have to have Scala installed or not. It turns out that having Scala installed natively is not necessary, seeing as the \texttt{scala-maven-plugin} is sufficient. 
  
  Steps to compile DBPedia Extraction Framework from the command line:
  \begin{enumerate}
   \item Ensure you have the JDK 1.8.x installed.
   \item Ensure Maven 3.x is installed.
   \item \texttt{mvn package} 
  \end{enumerate}

  Steps to compile DBPedia Extraction Framework from the Scala IDE (which can be downloaded from \href{http://scala-ide.org/}{Scala-ide.org}):
  \begin{enumerate}
   \item Ensure you have the JDK 1.8.x installed.
   \item Ensure you have the Scala IDE installed.
   \item \texttt{mvn eclipse:eclipse}
   \item \texttt{mvn package}
   \item Import existing Maven project into Scala IDE.
   \item Run \texttt{mvn clean install} from within the IDE.
  \end{enumerate}
  

  
  
  \bibliographystyle{amsplain}
  \bibliography{../../../BibliographicDetails_v.0.1}
 
\end{document}
