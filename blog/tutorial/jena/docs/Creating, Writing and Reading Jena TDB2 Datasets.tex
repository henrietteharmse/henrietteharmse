\documentclass{amsart}
%\documentclass[a4paper,10pt]{scrartcl}

\usepackage[utf8x]{inputenc}
\usepackage[british]{babel}
%\usepackage[a4paper, inner=0.5cm, outer=0.5cm, top=1cm,
%bottom=1.5cm, bindingoffset=1cm]{geometry}
\usepackage{amsmath}
\usepackage{amssymb, latexsym}
\usepackage{longtable}
\usepackage[table]{xcolor}
\usepackage{textcomp} 
\usepackage{stmaryrd}
\usepackage{graphicx}
\usepackage{enumitem}
\usepackage{yfonts}
\usepackage{algpseudocode}
\usepackage{algorithm}
\usepackage{hyperref}
\usepackage{MnSymbol}

\setlist[enumerate]{label*=\arabic*.}
\newtheorem{theorem}{Theorem}[section]
\newtheorem{example}{Example}[section]
\newtheorem{definition}{Definition}[section]
\newtheorem{proposition}{Proposition}[section]
\newtheorem{notation}{Notation}[section]

\renewcommand{\algorithmicrequire}{\textbf{Input:}}
\renewcommand{\algorithmicensure}{\textbf{Output:}}

\title{Creating, Writing and Reading Jena TDB2 Datasets}
\author{Henriette Harmse}
\date{\today}

\pdfinfo{%
  /Title    (Creating, Writing and Reading Jena TDB2 Datasets)
  /Author   (Henriette Harmse)
  /Creator  ()
  /Producer ()
  /Subject  (DL)
  /Keywords ()
}

\begin{document}
\maketitle
  Jena TDB2 can be used as an RDF datastore. Note that TDB (version 1 of Jena TDB) and TDB2 are not compatible with each other. 
  TDB2 is per definition transactional (while TDB is not). In this post I give a simple example that
  \begin{enumerate}
   \item create a new Jena TDB2 dataset,
   \item create a write transaction and write data to the datastore, 
   \item create a read transaction and read the data from the datastore, and
   \item release resources associated with the dataset on writing and reading is done.
  \end{enumerate}


  \section{Create TDB2 Dataset}
  To create a Jena TDB2 dataset, we use the \texttt{TDB2Factory}. Note that the class name is \texttt{TDB}\textbf{2}\texttt{Factory} and not \texttt{TDBFactory}. We need to specify a directory where our dataset will be created. Multiple datasets cannot be written to the same directory.
  
  \begin{small}
   \begin{verbatim}
Path path = Paths.get(".").toAbsolutePath().normalize();      
String dbDir = path.toFile().getAbsolutePath() + "/db/"; 
Location location = Location.create(dbDir);      
Dataset dataset = TDB2Factory.connectDataset(location);    
   \end{verbatim}
  \end{small}
  
  \section{Create WRITE Transaction and Write}
  \begin{small}
   \begin{verbatim}
dataset.begin(ReadWrite.WRITE);
UpdateRequest updateRequest = UpdateFactory.create(
  "INSERT DATA {<http://dbpedia.org/resource/Grace_Hopper> " 
  + "<http://xmlns.com/foaf/0.1/name> \"Grace Hopper\" .}");
UpdateProcessor updateProcessor = 
  UpdateExecutionFactory.create(updateRequest, dataset);
updateProcessor.execute();
dataset.commit();    
   \end{verbatim}
  \end{small}
  
  \section{Create READ Transaction and Read}
  \begin{small}
  \begin{verbatim}
dataset.begin(ReadWrite.READ);
QueryExecution qe = QueryExecutionFactory
  .create("SELECT ?s ?p ?o WHERE {?s ?p ?o .}", dataset);
for (ResultSet results = qe.execSelect(); results.hasNext();) {
  QuerySolution qs = results.next();
  String strValue = qs.get("?o").toString();
  logger.trace("value = " + strValue);
}   
  \end{verbatim} 
  \end{small}
  
  \section{Release Dataset Resources and Run Application}
  The dataset resources can be release calling \texttt{close()} on the dataset.  
  \begin{small}
   \begin{verbatim}
dataset.close();
   \end{verbatim}
  \end{small}
  
  Running the application will cause a \texttt{/db} directory to be create in the directory from where you run your application, which consists of the various files that represent your dataset. 
  
  \section{Conclusion}
  In this post I have given a simple example creating a TDB2 dataset and writing to and reading from it. This code can be found on \href{}{github}
  

  

  
  
  \bibliographystyle{amsplain}
  \bibliography{../../../BibliographicDetails_v.0.1}
 
\end{document}
