\documentclass{amsart}
%\documentclass[a4paper,10pt]{scrartcl}

\usepackage[utf8x]{inputenc}
\usepackage[british]{babel}
%\usepackage[a4paper, inner=0.5cm, outer=0.5cm, top=1cm,
%bottom=1.5cm, bindingoffset=1cm]{geometry}
\usepackage{amsmath}
\usepackage{amssymb, latexsym}
\usepackage{longtable}
\usepackage[table]{xcolor}
\usepackage{textcomp} 
\usepackage{stmaryrd}
\usepackage{graphicx}
\usepackage{enumitem}
\usepackage{yfonts}
\usepackage{algpseudocode}
\usepackage{algorithm}
\usepackage{hyperref}
\usepackage{MnSymbol}

\setlist[enumerate]{label*=\arabic*.}
\newtheorem{theorem}{Theorem}[section]
\newtheorem{example}{Example}[section]
\newtheorem{definition}{Definition}[section]
\newtheorem{proposition}{Proposition}[section]
\newtheorem{notation}{Notation}[section]

\renewcommand{\algorithmicrequire}{\textbf{Input:}}
\renewcommand{\algorithmicensure}{\textbf{Output:}}

\title{Creating a Remote Repository for GraphDB with RDF4J Programmatically}
\author{Henriette Harmse}
\date{\today}

\pdfinfo{%
  /Title    (Creating a Remote Repository for GraphDB with RDF4J Programmatically)
  /Author   (Henriette Harmse)
  /Creator  ()
  /Producer ()
  /Subject  (DL)
  /Keywords ()
}

\begin{document}
  \maketitle
  In my previous post I have detailed how you can create a local Ontotext GraphDB repository using RDF4J. I indicated that there are some problems when creating a local repository. Therefore, in this post I will detail how to create a remote Ontotext GraphDB repository using RDF4J. As with creating a local repository, there are three steps:
  \begin{enumerate}
    \item Create a configuration file, which is as for local repositories.
    \item Create \texttt{pom.xml} file, which is as for local repositories.
    \item Create the Java code.
  \end{enumerate}
  The benefit of creating a remote repository is that it will be under the control of the Ontotext GraphDB Workbench. Hence, you will be able to monitor your repository from the Workbench.
  
  \section{Java Code}
  \begin{small}
   \begin{verbatim}
package org.graphdb.rdf4j.tutorial;

import java.io.FileInputStream;
import java.io.InputStream;
import java.nio.file.Path;
import java.nio.file.Paths;
import java.util.Iterator;

import org.eclipse.rdf4j.model.Model;
import org.eclipse.rdf4j.model.Resource;
import org.eclipse.rdf4j.model.Statement;
import org.eclipse.rdf4j.model.impl.TreeModel;
import org.eclipse.rdf4j.model.util.Models;
import org.eclipse.rdf4j.model.vocabulary.RDF;
import org.eclipse.rdf4j.repository.Repository;
import org.eclipse.rdf4j.repository.RepositoryConnection;
import org.eclipse.rdf4j.repository.config.RepositoryConfig;
import org.eclipse.rdf4j.repository.config.RepositoryConfigSchema;
import org.eclipse.rdf4j.repository.http.config.HTTPRepositoryConfig;
import org.eclipse.rdf4j.repository.manager.RemoteRepositoryManager;
import org.eclipse.rdf4j.repository.manager.RepositoryManager;
import org.eclipse.rdf4j.repository.manager.RepositoryProvider;
import org.eclipse.rdf4j.rio.RDFFormat;
import org.eclipse.rdf4j.rio.RDFParser;
import org.eclipse.rdf4j.rio.Rio;
import org.eclipse.rdf4j.rio.helpers.StatementCollector;
import org.slf4j.Logger;
import org.slf4j.LoggerFactory;
import org.slf4j.Marker;
import org.slf4j.MarkerFactory;

public class CreateRemoteRepository {
  private static Logger logger = LoggerFactory.getLogger(CreateRemoteRepository.class);
  // Why This Failure marker
  private static final Marker WTF_MARKER = MarkerFactory.getMarker("WTF");
	
  public static void main(String[] args) {
    try {		
      Path path = Paths.get(".").toAbsolutePath().normalize();
      String strRepositoryConfig = path.toFile().getAbsolutePath() + "/src/main/resources/repo-defaults.ttl";
      String strServerUrl = "http://localhost:7200";
		
      // Instantiate a local repository manager and initialize it
      RepositoryManager repositoryManager  = RepositoryProvider.getRepositoryManager(strServerUrl);
      repositoryManager.initialize();
      repositoryManager.getAllRepositories();

      // Instantiate a repository graph model
      TreeModel graph = new TreeModel();

      // Read repository configuration file
      InputStream config = new FileInputStream(strRepositoryConfig);
      RDFParser rdfParser = Rio.createParser(RDFFormat.TURTLE);
      rdfParser.setRDFHandler(new StatementCollector(graph));
      rdfParser.parse(config, RepositoryConfigSchema.NAMESPACE);
      config.close();

      // Retrieve the repository node as a resource
      Resource repositoryNode =  Models.subject(graph
        .filter(null, RDF.TYPE, RepositoryConfigSchema.REPOSITORY))
        .orElseThrow(() -> new RuntimeException(
            "Oops, no <http://www.openrdf.org/config/repository#> subject found!"));

		
      // Create a repository configuration object and add it to the repositoryManager		
      RepositoryConfig repositoryConfig = RepositoryConfig.create(graph, repositoryNode);
      repositoryManager.addRepositoryConfig(repositoryConfig);

      // Get the repository from repository manager, note the repository id 
      // set in configuration .ttl file
      Repository repository = repositoryManager.getRepository("graphdb-repo");

      // Open a connection to this repository
      RepositoryConnection repositoryConnection = repository.getConnection();

      // ... use the repository

      // Shutdown connection, repository and manager
      repositoryConnection.close();
      repository.shutDown();
      repositoryManager.shutDown();					
   } catch (Throwable t) {
     logger.error(WTF_MARKER, t.getMessage(), t);
   }		
  }
}    
   \end{verbatim}
  \end{small}

  
  \section{Conclusion}
  In this post I detailed how you can create remote repository for Ontotext GraphDB using RDF4J, as well as the benefit of creating a remote repository rather than a local repository. You can find the complete code of this example on \href{https://github.com/henrietteharmse/henrietteharmse/tree/master/blog/tutorial/graphdb/source/rdf4j}{github}.
  
  \bibliographystyle{amsplain}
  \bibliography{../../../BibliographicDetails_v.0.1}
 
\end{document}
