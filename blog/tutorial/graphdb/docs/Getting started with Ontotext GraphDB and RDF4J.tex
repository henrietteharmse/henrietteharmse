\documentclass{amsart}
%\documentclass[a4paper,10pt]{scrartcl}

\usepackage[utf8x]{inputenc}
\usepackage[british]{babel}
%\usepackage[a4paper, inner=0.5cm, outer=0.5cm, top=1cm,
%bottom=1.5cm, bindingoffset=1cm]{geometry}
\usepackage{amsmath}
\usepackage{amssymb, latexsym}
\usepackage{longtable}
\usepackage[table]{xcolor}
\usepackage{textcomp} 
\usepackage{stmaryrd}
\usepackage{graphicx}
\usepackage{enumitem}
\usepackage{yfonts}
\usepackage{algpseudocode}
\usepackage{algorithm}
\usepackage{hyperref}
\usepackage{MnSymbol}

\setlist[enumerate]{label*=\arabic*.}
\newtheorem{theorem}{Theorem}[section]
\newtheorem{example}{Example}[section]
\newtheorem{definition}{Definition}[section]
\newtheorem{proposition}{Proposition}[section]
\newtheorem{notation}{Notation}[section]

\renewcommand{\algorithmicrequire}{\textbf{Input:}}
\renewcommand{\algorithmicensure}{\textbf{Output:}}

\title{Getting started with Ontotext GraphDB and RDF4J}
\author{Henriette Harmse}
\date{\today}

\pdfinfo{%
  /Title    (Getting started with Ontotext GraphDB and RDF4J)
  /Author   (Henriette Harmse)
  /Creator  ()
  /Producer ()
  /Subject  (DL)
  /Keywords ()
}

\begin{document}
  \maketitle
  In this post I will explain how to quickly get started with the free version of Ontotext GraphDB and RDF4J. Ontotext GraphDB is an RDF datastore and RDF4J is a Java framework for accessing RDF datastores (not just GraphDB).
  I will explain 
  \begin{enumerate}
   \item how to install and start GraphDB, as well as how to use the workbench to add a repository, and
   \item how to do SPARQL queries against GraphDB using RDF4J.
  \end{enumerate}

  \section{Install and start GraphDB and create a Repository}
  To gain access  to the free version of GraphDB you have to email Ontotext. They will respond with an email with links to a desktop and stand-alone server version of GraphDB. You want to download the stand-alone server version. This is a \texttt{graphdb-free-VERSION-dist.zip} file, that you can extract somewhere on your filesystem, which I will refer to here as \texttt{\$GRAPHDB\_ROOT}. To start GraphDB, go to  \texttt{\$GRAPHDB\_ROOT/bin} and run \texttt{./graphdb}.
  
  To access the workbench you can go to \texttt{http://localhost:7200}. To create a new repository, in the left-hand side menu navigate to \texttt{Setup-->Repositories}.  Click the \texttt{Create new repository} button. For our simple example we will use \texttt{PersonData} as an \texttt{Repository ID}. The rest of the settings we leave as-is. At the bottom of the page you can press the \texttt{Create} button.
  
    
  \section{Accessing a GraphDB Repository using RDF4J}
  To access our \texttt{PersonData} repository we will use RDF4J.  Since GraphDB is based on the RDF4J libraries, we only need to include the GraphDB dependencies since these already include RDF4J. Thus, in our \texttt{pom.xml} file we only need to add the following:

\begin{small}
\begin{verbatim} 
<dependency>
    <groupId>com.ontotext.graphdb</groupId>
    <artifactId>graphdb-free-runtime</artifactId>
    <version>8.5.0</version>
</dependency>
\end{verbatim}
\end{small}

In our example Java code we first insert some data and then do a query based on the added data. For inserting data we start a transaction and commit it, or, if it fails we do a rollback. For querying the data we iterate through the \texttt{TupleQueryResult}, retrieving values for the binding variables we are interested in (i.e. \texttt{name} in this case). In line with the \texttt{TupleQueryResult} documentation, we close the \texttt{TupleQueryResult} once we are done.

\begin{small}
\begin{verbatim} 
package org.graphdb.rdf4j.tutorial;

import org.eclipse.rdf4j.model.impl.SimpleLiteral;
import org.eclipse.rdf4j.query.BindingSet;
import org.eclipse.rdf4j.query.QueryEvaluationException;
import org.eclipse.rdf4j.query.QueryLanguage;
import org.eclipse.rdf4j.query.TupleQuery;
import org.eclipse.rdf4j.query.TupleQueryResult;
import org.eclipse.rdf4j.query.Update;
import org.eclipse.rdf4j.repository.Repository;
import org.eclipse.rdf4j.repository.RepositoryConnection;
import org.eclipse.rdf4j.repository.http.HTTPRepository;
import org.slf4j.Logger;
import org.slf4j.LoggerFactory;
import org.slf4j.Marker;
import org.slf4j.MarkerFactory;

public class SimpleInsertQueryExample {
  private static Logger logger = LoggerFactory.getLogger(SimpleInsertQueryExample.class);
  // Why This Failure marker
  private static final Marker WTF_MARKER = MarkerFactory.getMarker("WTF");
  
  // GraphDB 
  private static final String GRAPHDB_SERVER = "http://localhost:7200/";
  private static final String REPOSITORY_ID = "PersonData";


  private static String strInsert;
  private static String strQuery;
  
  static {
    strInsert = 
        "INSERT DATA {"
         + "<http://dbpedia.org/resource/Grace_Hopper> <http://dbpedia.org/ontology/birthDate> \"1906-12-09\"^^<http://www.w3.org/2001/XMLSchema#date> ."
         + "<http://dbpedia.org/resource/Grace_Hopper> <http://dbpedia.org/ontology/birthPlace> <http://dbpedia.org/resource/New_York_City> ."
         + "<http://dbpedia.org/resource/Grace_Hopper> <http://dbpedia.org/ontology/deathDate> \"1992-01-01\"^^<http://www.w3.org/2001/XMLSchema#date> ."
         + "<http://dbpedia.org/resource/Grace_Hopper> <http://dbpedia.org/ontology/deathPlace> <http://dbpedia.org/resource/Arlington_County,_Virginia> ."
         + "<http://dbpedia.org/resource/Grace_Hopper> <http://purl.org/dc/terms/description> \"American computer scientist and United States Navy officer.\" ."
         + "<http://dbpedia.org/resource/Grace_Hopper> <http://www.w3.org/1999/02/22-rdf-syntax-ns#type> <http://dbpedia.org/ontology/Person> ."
         + "<http://dbpedia.org/resource/Grace_Hopper> <http://xmlns.com/foaf/0.1/gender> \"female\" ."
         + "<http://dbpedia.org/resource/Grace_Hopper> <http://xmlns.com/foaf/0.1/givenName> \"Grace\" ."
         + "<http://dbpedia.org/resource/Grace_Hopper> <http://xmlns.com/foaf/0.1/name> \"Grace Hopper\" ."
         + "<http://dbpedia.org/resource/Grace_Hopper> <http://xmlns.com/foaf/0.1/surname> \"Hopper\" ."        
         + "}";
    
    strQuery = 
        "SELECT ?name FROM DEFAULT WHERE {" +
        "?s <http://xmlns.com/foaf/0.1/name> ?name .}";
  }  
  
  private static RepositoryConnection getRepositoryConnection() {
    Repository repository = new HTTPRepository(GRAPHDB_SERVER, REPOSITORY_ID);
    repository.initialize();
    RepositoryConnection repositoryConnection = repository.getConnection();
    return repositoryConnection;
  }
  
  private static void insert(RepositoryConnection repositoryConnection) {
    repositoryConnection.begin();
    
    Update updateOperation = repositoryConnection.prepareUpdate(QueryLanguage.SPARQL, strInsert);
    updateOperation.execute();
    
    try {
      repositoryConnection.commit();
    } catch (Exception e) {
      if (repositoryConnection.isActive())
        repositoryConnection.rollback();
    }
  }

  private static void query(RepositoryConnection repositoryConnection) {
    TupleQuery tupleQuery = repositoryConnection.prepareTupleQuery(QueryLanguage.SPARQL, strQuery);
    TupleQueryResult result = null;
    try {
      result = tupleQuery.evaluate();
      while (result.hasNext()) {
        BindingSet bindingSet = result.next();

        SimpleLiteral name = (SimpleLiteral)bindingSet.getValue("name");
        logger.trace("name = " + name.stringValue());
      }
    }
    catch (QueryEvaluationException qee) {
      logger.error(WTF_MARKER, qee.getStackTrace().toString(), qee);
    } finally {
      result.close();
    }    
  }  
  
  public static void main(String[] args) {
    RepositoryConnection repositoryConnection = null;
    try {   
      repositoryConnection = getRepositoryConnection();
      
      insert(repositoryConnection);
      query(repositoryConnection);      
      
    } catch (Throwable t) {
      logger.error(WTF_MARKER, t.getMessage(), t);
    } finally {
      repositoryConnection.close();
    }
  }  
}
\end{verbatim}
\end{small}

\section{Conclusion}
    In this brief post I gave a quick example of how you can setup a simple GraphDB repository and query it using SPARQL. You can find sample code on
\href{https://github.com/henrietteharmse/henrietteharmse/tree/master/blog/tutorial/graphdb/source/rdf4j}{github}.
  
  \bibliographystyle{amsplain}
  \bibliography{../../../BibliographicDetails_v.0.1}
 
\end{document}
